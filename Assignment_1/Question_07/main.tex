%%%%%%%%%%%%%%%%%%%%%%%%%%%%%%%%%%%%%%%%%%%%%%%%%%%%%%%%%%%%%%%%%%
%%%%%%%%%%%%%%%%%%%%%%%%%%%%%%%%%%%%%%%%%%%%%%%%%%%%%%%%%%%%%%%%%%
% \setmathfont{TeX Gyre Termes Math}
%Packages
\documentclass[10pt, a4paper]{article}
\usepackage[top=3cm, bottom=4cm, left=2cm, right=2cm]{geometry}
\usepackage{amsmath,amsthm,amsfonts,amssymb,amscd, fancyhdr, color, comment, graphicx, environ}
\usepackage{float}
\usepackage{booktabs}
\usepackage{pifont}
\usepackage{mathrsfs}
\usepackage[math-style=ISO]{unicode-math}
\usepackage{lastpage}
\usepackage[dvipsnames]{xcolor}
\usepackage[framemethod=TikZ]{mdframed}
\usepackage{enumerate}
\usepackage[shortlabels]{enumitem}
\usepackage{fancyhdr}
\usepackage{indentfirst}
\usepackage{listings}
\usepackage{sectsty}
\usepackage{thmtools}
\usepackage{shadethm}
\usepackage{hyperref}
\usepackage{setspace}
\usepackage{adjustbox}
\hypersetup{
    colorlinks=true,
    linkcolor=blue,
    filecolor=magenta,      
    urlcolor=blue,
}
\usepackage{xcolor,colortbl}
\usepackage{listings}
\usepackage{xcolor}

\lstset{ 
    language=bash,
    basicstyle=\ttfamily\small,
    numbers=left,
    numberstyle=\tiny\color{gray},
    stepnumber=1,
    numbersep=5pt,
    backgroundcolor=\color{lightgray},
    showspaces=false,
    showstringspaces=false,
    showtabs=false,
    frame=single,
    rulecolor=\color{black},
    captionpos=b,
    breaklines=true,
    breakatwhitespace=false,
    escapeinside={\%*}{*)}
}
%%%%%%%%%%%%%%%%%%%%%%%%%%%%%%%%%%%%%%%%%%%%%%%%%%%%%%%%%%%%%%%%%%
%%%%%%%%%%%%%%%%%%%%%%%%%%%%%%%%%%%%%%%%%%%%%%%%%%%%%%%%%%%%%%%%%%
%Environment setup
\mdfsetup{skipabove=\topskip,skipbelow=\topskip}
\newrobustcmd\ExampleText{%
An \textit{inhomogeneous linear} differential equation has the form
\begin{align}
L[v ] = f,
\end{align}
where $L$ is a linear differential operator, $v$ is the dependent
variable, and $f$ is a given non−zero function of the independent
variables alone.
}
\mdfdefinestyle{theoremstyle}{%
linecolor=black,linewidth=1pt,%
frametitlerule=true,%
frametitlebackgroundcolor=gray!20,
innertopmargin=\topskip,
}
\mdtheorem[style=theoremstyle]{Problem}{Question Number}
\setcounter{Problem}{6}
\newenvironment{Solution}{\textbf{Solution.}}

\definecolor{codegreen}{rgb}{0,0.6,0}
\definecolor{codegray}{rgb}{0.5,0.5,0.5}
\definecolor{codepurple}{rgb}{0.58,0,0.82}
\definecolor{backcolour}{rgb}{0.95,0.95,0.92}

\lstdefinestyle{mystyle}{
    backgroundcolor=\color{backcolour},   
    commentstyle=\color{codegreen},
    keywordstyle=\color{magenta},
    numberstyle=\tiny\color{codegray},
    stringstyle=\color{codepurple},
    basicstyle=\ttfamily\footnotesize,
    breakatwhitespace=false,         
    breaklines=true,                 
    captionpos=b,                    
    keepspaces=true,                 
    numbers=false,                    
    numbersep=5pt,                  
    showspaces=false,                
    showstringspaces=false,
    showtabs=false,                  
    tabsize=2
}

\lstset{style=mystyle}
%%%%%%%%%%%%%%%%%%%%%%%%%%%%%%%%%%%%%%%%%%%%%%%%%%%%%%%%%%%%%%%%%%
%%%%%%%%%%%%%%%%%%%%%%%%%%%%%%%%%%%%%%%%%%%%%%%%%%%%%%%%%%%%%%%%%%
%Fill in the appropriate information below
\newcommand{\norm}[1]{\left\lVert#1\right\rVert}     
\newcommand\course{XXXX0000}                            % <-- course name   
\newcommand\hwnumber{0}                                 % <-- homework number
\newcommand\Information{Someone}                        % <-- personal information
%%%%%%%%%%%%%%%%%%%%%%%%%%%%%%%%%%%%%%%%%%%%%%%%%%%%%%%%%%%%%%%%%%
%%%%%%%%%%%%%%%%%%%%%%%%%%%%%%%%%%%%%%%%%%%%%%%%%%%%%%%%%%%%%%%%%%
%Page setup
\pagestyle{fancy}
\headheight 35pt
\lhead{\today \hspace*{4cm} Key-Breakers}
\rhead{\includegraphics[width=1.2cm]{../logo.png}}
\lfoot{}
\pagenumbering{arabic}
\cfoot{\small\thepage}
\rfoot{}
\headsep 1.2em
\renewcommand{\baselinestretch}{1.25}
%%%%%%%%%%%%%%%%%%%%%%%%%%%%%%%%%%%%%%%%%%%%%%%%%%%%%%%%%%%%%%%%%%
%%%%%%%%%%%%%%%%%%%%%%%%%%%%%%%%%%%%%%%%%%%%%%%%%%%%%%%%%%%%%%%%%%
%Add new commands here
\renewcommand{\labelenumi}{\alph{enumi})}
\newcommand{\Z}{\mathbb Z}
\newcommand{\R}{\mathbb R}
\newcommand{\Q}{\mathbb Q}
\newcommand{\NN}{\mathbb N}
\newcommand{\PP}{\mathbb P}
\DeclareMathOperator{\Mod}{Mod} 
\renewcommand\lstlistingname{Algorithm}
\renewcommand\lstlistlistingname{Algorithms}
\def\lstlistingautorefname{Alg.}
\newtheorem*{theorem}{Theorem}
\newtheorem*{lemma}{Lemma}
\newtheorem{case}{Case}
\newcommand{\assign}{:=}
\newcommand{\infixiff}{\text{ iff }}
\newcommand{\nobracket}{}
\newcommand{\backassign}{=:}
\newcommand{\tmmathbf}[1]{\ensuremath{\boldsymbol{#1}}}
\newcommand{\tmop}[1]{\ensuremath{\operatorname{#1}}}
\newcommand{\tmtextbf}[1]{\text{{\bfseries{#1}}}}
\newcommand{\tmtextit}[1]{\text{{\itshape{#1}}}}

\newenvironment{itemizedot}{
    \begin{itemize} 
    \renewcommand{\labelitemi}{$\bullet$}
    \renewcommand{\labelitemii}{$\bullet$}
    \renewcommand{\labelitemiii}{$\bullet$}
    \renewcommand{\labelitemiv}{$\bullet$}}
    {\end{itemize}}

\catcode`\<=\active\def<{
\fontencoding{T1}\selectfont\symbol{60}\fontencoding{\encodingdefault}}
\catcode`\>=\active\def>{
\fontencoding{T1}\selectfont\symbol{62}\fontencoding{\encodingdefault}}
\catcode`\<=\active\def<{
\fontencoding{T1}\selectfont\symbol{60}\fontencoding{\encodingdefault}}

%%%%%%%%%%%%%%%%%%%%%%%%%%%%%%%%%%%%%%%%%%%%%%%%%%%%%%%%%%%%%%%%%%
%%%%%%%%%%%%%%%%%%%%%%%%%%%%%%%%%%%%%%%%%%%%%%%%%%%%%%%%%%%%%%%%%%
%Begin now!



\begin{document}
%%%%%%%%%%%%%%%%%%%%%%%%%%%%%%%%%%%%%%%%%%%%%%%%%%%%%%%%%%%%%%%%%%
%%%%%%%%%%%%%%%%%%%%%%%%%%%%%%%%%%%%%%%%%%%%%%%%%%%%%%%%%%%%%%%%%%
%Start the assignment now
%%%%%%%%%%%%%%%%%%%%%%%%%%%%%%%%%%%%%%%%%%%%%%%%%%%%%%%%%%%%%%%%%%
%New problem
\newpage
\begin{Problem}
    Recall the Vigenre Cipher discussed in class.
    \begin{itemize}
        \item  Encrypt MY HOME IS INDIA using a Vigenre Cipher with keyword KEA.
        \item The ciphertext JNHYSMCDJOP was obtained by encrypting an English word using a
              Vigenre Cipher:
              \begin{itemize}
                  \item The first and ninth letters of the plaintext are identical. What does this tell you
                        about the key?
                  \item Given the keyword length is less than 7, the third and fourth letters of the plaintext
                        are F and O, respectively, and A is a letter in the keyword, find the keyword and the
                        plaintext.
              \end{itemize}
        \item
              If the keyword of a Vigenre Cipher has repeated letters, does this make it any easier to
              break?
    \end{itemize}
\end{Problem}

\begin{Solution}
    \subsection*{Part1}
    \textbf{Vigenre Cipher Table}
    \begin{table}[h!]
        \centering
        \renewcommand{\arraystretch}{1.5} % Adjust the row height
        \begin{tabular}{|c|c|c|c|c|c|c|c|c|c|c|c|c|c|}
            \hline
            \textbf{Plain Text}  & M    & Y    & H   & O    & M    & E   & I    & S    & I   & N    & D   & I   & A    \\
            \textbf{}            & (12) & (24) & (7) & (14) & (12) & (4) & (8)  & (18) & (8) & (13) & (3) & (8) & (0)  \\
            \hline
            \textbf{Key}         & K    & E    & A   & K    & E    & A   & K    & E    & A   & K    & E   & A   & K    \\
            \textbf{}            & (10) & (4)  & (0) & (10) & (4)  & (0) & (10) & (4)  & (0) & (10) & (4) & (0) & (10) \\
            \hline
            \textbf{Cipher Text} & W    & C    & H   & Y    & Q    & E   & S    & W    & I   & X    & H   & I   & K    \\
            \textbf{}            & (22) & (2)  & (7) & (24) & (16) & (4) & (18) & (22) & (8) & (23) & (7) & (8) & (10) \\
            \hline
        \end{tabular}
        \caption{Vigenre Cipher}
        \label{tab:vigenre}
    \end{table}

    \pagebreak
    \subsection*{Part2.1}
    Given the ciphertext \texttt{JNHYSMCDJOP} obtained by encrypting an English word using a Vigenre cipher, \\
    and knowing that the first and ninth letters of the plaintext are identical, we can infer the following \\
    about the key: \\

    \textbf{Key Points}
    \begin{enumerate}
        \item \textbf{Plaintext Repetition:} \\
              The first and ninth letters of the plaintext are the same,  \( P_1 = P_9 \).

        \item \textbf{Ciphertext Repetition:} \\
              The ciphertext letters corresponding to these positions are also identical,  \( C_1 = C_9 \).

              Since \( P_1 = P_9 \) and \( C_1 = C_9 \), it follows that:
              \[
                  (P_1 + K_1) \mod 26 = (P_9 + K_9) \mod 26
              \]
              Given \( P_1 = P_9 \), it simplifies to:
              \[
                  K_1 \equiv K_9 \pmod{26}
              \]

        \item The equality \( K_1 = K_9 \) indicates that the key must repeat every 8
              characters or a divisor of 8 (1, 2, 4, 8) because the positions 1 and 9 are 8 characters apart.

        \item If the key length is shorter than 8 characters, it must be a repeating key whose
              length divides 8. Thus, the key could be of lengths 1, 2, 4, or 8.

        \item So we can conclued from observation that \( P_1 = P_9 \) and \( C_1 = C_9 \) provides a constraint
              on the key's periodicity. The key must repeat at intervals that align with the distance between these
              positions in the plaintext, giving us valuable insight into possible key lengths and their periodic patterns.
    \end{enumerate}

    \subsection*{Part2.2}
    Given

    \begin{itemize}
        \item We Know key length is less than 7
        \item Also given thrid and fourth letters of plain text as 'F' and 'O'.
        \item Key contain letter 'A'
    \end{itemize}
    From above part we know that if key length is less than 8 then possible length
    of key is 1,2 and 4. So we can try all key of length 1,2 and 4 which contain
    letter 'A'.\\

    \textbf{Calculating the key for the 3rd and 4th positions}\\
    \begin{enumerate}
        \item For the 3rd position:
              \begin{itemize}
                  \item \(C_3 = H\) (ciphertext) and \(P_3 = F\) (plaintext)
              \end{itemize}

              Calculate the key:
              \[
                  K_3 = (C_3 - P_3) \mod 26 = (H - F) \mod 26 = (7 - 5) \mod 26 = 2
              \]
              So, \(K_3\) corresponds to the letter \textbf{C}.
        \item For the 4th position:
              \begin{itemize}
                  \item \(C_4 = Y\) (ciphertext) and \(P_4 = O\) (plaintext)
              \end{itemize}
              Calculate the key:
              \[ K_4 = (C_4 - P_4) \mod 26 = (Y - O) \mod 26 = (24 - 14) \mod 26 = 10 \]
              So, \(K_4\) corresponds to the letter \textbf{K}.
    \end{enumerate}

    From above result we can conclude that our key Contains letter A, C and K. Also we can
    discard the posibility of key length 1,2. So our key lenght is 4. Our key is in the following
    format
    \begin{enumerate}[label=\roman*]
        \item *ACK
        \item A*CK
    \end{enumerate}

    We can try all possible keys to decrypt the given cipher text to
    plain text and find valid english word.
    To achieve this we use python code(\texttt{vigenre.py}) to iteratively eliminate invalid keys
    and find the correct one.\\

    \textbf{Note} Before executing \texttt{vigenre.py} file make sure you have installed all requried libraries listed in \\
    \texttt{requirements.txt} using following command.

    \begin{lstlisting}[]
        pip install -r requirements.txt
    \end{lstlisting}

    After Executing Code We will get Plaintext and Key as follows
    \begin{itemize}
        \item Plaintext  -> \textbf{INFORMATION}
        \item Key  -> \textbf{BACK}
    \end{itemize}

    \pagebreak

    \subsection*{Part3}
    Yes, the presence of repeated letters in a Vigenère Cipher's keyword can indeed make it more susceptible to decryption. When a keyword contains repetitions, it introduces periodic patterns in the ciphertext. These patterns can be detected and analyzed using frequency analysis techniques. For example, if the keyword is repeating, the cipher effectively transforms into a series of Caesar ciphers, each with a shift determined by the corresponding letter of the keyword.

    By identifying the repeating nature of the key, one can use methods such as the Kasiski examination to determine the length of the keyword. Once the length is known, the ciphertext can be segmented into multiple Caesar ciphers, each corresponding to a single position in the keyword. Analyzing the frequency of letters in these segments can then help in deducing the individual letters of the repeating key, leading to the eventual decryption of the plaintext.
\end{Solution}

%%%%%%%%%%%%%%%%%%%%%%%%%%%%%%%%%%%%%%%%%%%%%%%%%%%%%%%%%%%%%%%%%%
\end{document}

%%%%%%%%%%%%%%%%%%%%%%%%%%%%%%%%%%%%%%%%%%%%%%%%%%%%%%%%%%%%%%%%%%
%%%%%%%%%%%%%%%%%%%%%%%%%%%%%%%%%%%%%%%%%%%%%%%%%%%%%%%%%%%%%%%%%%