%%%%%%%%%%%%%%%%%%%%%%%%%%%%%%%%%%%%%%%%%%%%%%%%%%%%%%%%%%%%%%%%%%
%%%%%%%%%%%%%%%%%%%%%%%%%%%%%%%%%%%%%%%%%%%%%%%%%%%%%%%%%%%%%%%%%%
% \setmathfont{TeX Gyre Termes Math}
%Packages
\documentclass[10pt, a4paper]{article}
\usepackage[top=3cm, bottom=4cm, left=2cm, right=2cm]{geometry}
\usepackage{amsmath,amsthm,amsfonts,amssymb,amscd, fancyhdr, color, comment, graphicx, environ}
\usepackage{float}
\usepackage{booktabs}
\usepackage{pifont}
\usepackage{mathrsfs}
\usepackage[math-style=ISO]{unicode-math}
\usepackage{lastpage}
\usepackage[dvipsnames]{xcolor}
\usepackage[framemethod=TikZ]{mdframed}
\usepackage{enumerate}
\usepackage[shortlabels]{enumitem}
\usepackage{fancyhdr}
\usepackage{indentfirst}
\usepackage{listings}
\usepackage{sectsty}
\usepackage{thmtools}
\usepackage{shadethm}
\usepackage{hyperref}
\usepackage{setspace}
\usepackage{adjustbox}
\hypersetup{
    colorlinks=true,
    linkcolor=blue,
    filecolor=magenta,      
    urlcolor=blue,
}
\usepackage{xcolor,colortbl}
%%%%%%%%%%%%%%%%%%%%%%%%%%%%%%%%%%%%%%%%%%%%%%%%%%%%%%%%%%%%%%%%%%
%%%%%%%%%%%%%%%%%%%%%%%%%%%%%%%%%%%%%%%%%%%%%%%%%%%%%%%%%%%%%%%%%%
%Environment setup
\mdfsetup{skipabove=\topskip,skipbelow=\topskip}
\newrobustcmd\ExampleText{%
An \textit{inhomogeneous linear} differential equation has the form
\begin{align}
L[v ] = f,
\end{align}
where $L$ is a linear differential operator, $v$ is the dependent
variable, and $f$ is a given non−zero function of the independent
variables alone.
}
\mdfdefinestyle{theoremstyle}{%
linecolor=black,linewidth=1pt,%
frametitlerule=true,%
frametitlebackgroundcolor=gray!20,
innertopmargin=\topskip,
}
\mdtheorem[style=theoremstyle]{Problem}{Question Number}
\setcounter{Problem}{3}
\newenvironment{Solution}{\textbf{Solution.}}

\definecolor{codegreen}{rgb}{0,0.6,0}
\definecolor{codegray}{rgb}{0.5,0.5,0.5}
\definecolor{codepurple}{rgb}{0.58,0,0.82}
\definecolor{backcolour}{rgb}{0.95,0.95,0.92}

\lstdefinestyle{mystyle}{
    backgroundcolor=\color{backcolour},   
    commentstyle=\color{codegreen},
    keywordstyle=\color{magenta},
    numberstyle=\tiny\color{codegray},
    stringstyle=\color{codepurple},
    basicstyle=\ttfamily\footnotesize,
    breakatwhitespace=false,         
    breaklines=true,                 
    captionpos=b,                    
    keepspaces=true,                 
    numbers=left,                    
    numbersep=5pt,                  
    showspaces=false,                
    showstringspaces=false,
    showtabs=false,                  
    tabsize=2
}

\lstset{style=mystyle}
%%%%%%%%%%%%%%%%%%%%%%%%%%%%%%%%%%%%%%%%%%%%%%%%%%%%%%%%%%%%%%%%%%
%%%%%%%%%%%%%%%%%%%%%%%%%%%%%%%%%%%%%%%%%%%%%%%%%%%%%%%%%%%%%%%%%%
%Fill in the appropriate information below
\newcommand{\norm}[1]{\left\lVert#1\right\rVert}     
\newcommand\course{XXXX0000}                            % <-- course name   
\newcommand\hwnumber{0}                                 % <-- homework number
\newcommand\Information{Someone}                        % <-- personal information
%%%%%%%%%%%%%%%%%%%%%%%%%%%%%%%%%%%%%%%%%%%%%%%%%%%%%%%%%%%%%%%%%%
%%%%%%%%%%%%%%%%%%%%%%%%%%%%%%%%%%%%%%%%%%%%%%%%%%%%%%%%%%%%%%%%%%
%Page setup
\pagestyle{fancy}
\headheight 35pt
\lhead{\today \hspace*{4cm} Key-Breakers}
\rhead{\includegraphics[width=1.2cm]{../logo.png}}
\lfoot{}
\pagenumbering{arabic}
\cfoot{\small\thepage}
\rfoot{}
\headsep 1.2em
\renewcommand{\baselinestretch}{1.25}
%%%%%%%%%%%%%%%%%%%%%%%%%%%%%%%%%%%%%%%%%%%%%%%%%%%%%%%%%%%%%%%%%%
%%%%%%%%%%%%%%%%%%%%%%%%%%%%%%%%%%%%%%%%%%%%%%%%%%%%%%%%%%%%%%%%%%
%Add new commands here
\renewcommand{\labelenumi}{\alph{enumi})}
\newcommand{\Z}{\mathbb Z}
\newcommand{\R}{\mathbb R}
\newcommand{\Q}{\mathbb Q}
\newcommand{\NN}{\mathbb N}
\newcommand{\PP}{\mathbb P}
\DeclareMathOperator{\Mod}{Mod} 
\renewcommand\lstlistingname{Algorithm}
\renewcommand\lstlistlistingname{Algorithms}
\def\lstlistingautorefname{Alg.}
\newtheorem*{theorem}{Theorem}
\newtheorem*{lemma}{Lemma}
\newtheorem{case}{Case}
\newcommand{\assign}{:=}
\newcommand{\infixiff}{\text{ iff }}
\newcommand{\nobracket}{}
\newcommand{\backassign}{=:}
\newcommand{\tmmathbf}[1]{\ensuremath{\boldsymbol{#1}}}
\newcommand{\tmop}[1]{\ensuremath{\operatorname{#1}}}
\newcommand{\tmtextbf}[1]{\text{{\bfseries{#1}}}}
\newcommand{\tmtextit}[1]{\text{{\itshape{#1}}}}

\newenvironment{itemizedot}{
    \begin{itemize} 
    \renewcommand{\labelitemi}{$\bullet$}
    \renewcommand{\labelitemii}{$\bullet$}
    \renewcommand{\labelitemiii}{$\bullet$}
    \renewcommand{\labelitemiv}{$\bullet$}}
    {\end{itemize}}

\catcode`\<=\active\def<{
\fontencoding{T1}\selectfont\symbol{60}\fontencoding{\encodingdefault}}
\catcode`\>=\active\def>{
\fontencoding{T1}\selectfont\symbol{62}\fontencoding{\encodingdefault}}
\catcode`\<=\active\def<{
\fontencoding{T1}\selectfont\symbol{60}\fontencoding{\encodingdefault}}

%%%%%%%%%%%%%%%%%%%%%%%%%%%%%%%%%%%%%%%%%%%%%%%%%%%%%%%%%%%%%%%%%%
%%%%%%%%%%%%%%%%%%%%%%%%%%%%%%%%%%%%%%%%%%%%%%%%%%%%%%%%%%%%%%%%%%
%Begin now!



\begin{document}
%%%%%%%%%%%%%%%%%%%%%%%%%%%%%%%%%%%%%%%%%%%%%%%%%%%%%%%%%%%%%%%%%%
%%%%%%%%%%%%%%%%%%%%%%%%%%%%%%%%%%%%%%%%%%%%%%%%%%%%%%%%%%%%%%%%%%
%Start the assignment now
%%%%%%%%%%%%%%%%%%%%%%%%%%%%%%%%%%%%%%%%%%%%%%%%%%%%%%%%%%%%%%%%%%
%New problem
\newpage
\begin{Problem}
    An affine cryptosystem is given by the following encryption function, where $a, b$ are chosen
    from $\mathbb{Z}_{26}$.
    \begin{align*}
        enc_{a,b} : \mathbb{Z}_{26} \rightarrow \mathbb{Z}_{26} \\
        x \rightarrow $ax + b$    \epsilon    \mathbb{Z}_{26}
    \end{align*}

    \begin{itemize}
        \item Encrypt the plaintext cryptography using the affine code $enc_{3,5}$. What is the decryption
              function corresponding to $enc_{3,5}$? Decrypt the ciphertext XRHLAFUUK.
        \item A central requirement of cryptography is that the plaintext must be computable from the \\
              key and the ciphertext. Explain why $enc_{2,3}$ violates this rule. Show that the function \\
              enca,b satisfies the rule if and only if $gcd(a, 26) = 1$.
        \item
              In the following we consider only functions enca,b with $gcd(a, 26) = 1$. Show that all affine
              codes with $b = 0$ map the letter $a$ to a and the letter $n$ to n.
    \end{itemize}
\end{Problem}

\begin{Solution}

    \textbf{Part 1: Encryption and Decryption}

    We are given the plaintext \textit{cryptography} and the affine cipher parameters $a = 3$ and $b = 5$. The encryption function is given by:
    \[
    \text{enc}_{3,5}(x) = 3x + 5 \mod 26
    \]
    The decryption function can be derived by finding the modular inverse of $a$ modulo 26. Let's calculate it using the Python code provided.

    \begin{lstlisting}[language=Python, caption=Affine Cipher Encryption and Decryption]
def modinv(a, m):
    for x in range(1, m):
        if (a * x) % m == 1:
            return x
    return None

def encrypt(mssge,a,b):
    c = ""
    for i in mssge:
        if i.isalpha():
            if i.islower():
                c += chr(((a*(ord(i)-97)+b)%26)+97)
            else:
                c += chr(((a*(ord(i)-65)+b)%26)+65)
        else:
            c += i
    return c

def decrypt(cipher, a, b):
    mssge = ""
    for i in cipher:
        if i.isalpha():
            if i.islower():
                mssge += chr(((modinv(a, 26)*(ord(i)-97-b))%26)+97)
            else:
                mssge += chr(((modinv(a, 26)*(ord(i)-65-b))%26)+65)
        else:
            mssge += i
    return mssge

cipher = encrypt("cryptography", 3, 5)
print(cipher)
print(decrypt("XRHLAFUUK", 3, 5))
    \end{lstlisting}

    The ciphertext for \textit{cryptography} using $a = 3$ and $b = 5$ is \textbf{XRHLAFUUK}. The corresponding decryption function yields the original plaintext when applied to the ciphertext.

    \textbf{Part 2: Why does $enc_{2,3}$ violate the rule?}

    A central requirement in cryptography is that the plaintext must be computable from the key and the ciphertext. For $enc_{2,3}$, the encryption function is:
    \[
    \text{enc}_{2,3}(x) = 2x + 3 \mod 26
    \]
    However, the key $a = 2$ is not invertible modulo 26 because $\gcd(2, 26) \neq 1$. Specifically, $\gcd(2, 26) = 2$, which means there is no unique inverse for 2 modulo 26, and therefore, decryption is not guaranteed to retrieve the original plaintext. This violates the requirement that the plaintext should be retrievable from the ciphertext and the key.

    \textbf{Part 3: All affine codes with $b = 0$ map the letter $a$ to $a$ and the letter $n$ to $n$.}

    When $b = 0$, the affine encryption function simplifies to:
    \[
    \text{enc}_{a,0}(x) = ax \mod 26
    \]
    If we take $x = 0$ (which corresponds to the letter 'a'), then:
    \[
    \text{enc}_{a,0}(0) = 0 \mod 26
    \]
    which means 'a' is mapped to 'a'.

    Similarly, for $x = 13$ (which corresponds to the letter 'n'):
    \[
    \text{enc}_{a,0}(13) = 13a \mod 26
    \]
    Since $13a \mod 26 = 13$ (as long as $a$ is odd and $\gcd(a, 26) = 1$), the letter 'n' is mapped to 'n'.
\end{Solution}

%%%%%%%%%%%%%%%%%%%%%%%%%%%%%%%%%%%%%%%%%%%%%%%%%%%%%%%%%%%%%%%%%%
%Complete the assignment now
\end{document}

%%%%%%%%%%%%%%%%%%%%%%%%%%%%%%%%%%%%%%%%%%%%%%%%%%%%%%%%%%%%%%%%%%
%%%%%%%%%%%%%%%%%%%%%%%%%%%%%%%%%%%%%%%%%%%%%%%%%%%%%%%%%%%%%%%%%%