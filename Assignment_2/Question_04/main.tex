%%%%%%%%%%%%%%%%%%%%%%%%%%%%%%%%%%%%%%%%%%%%%%%%%%%%%%%%%%%%%%%%%%
%%%%%%%%%%%%%%%%%%%%%%%%%%%%%%%%%%%%%%%%%%%%%%%%%%%%%%%%%%%%%%%%%%
% \setmathfont{TeX Gyre Termes Math}
%Packages
\documentclass[10pt, a4paper]{article}
\usepackage[top=3cm, bottom=4cm, left=2cm, right=2cm]{geometry}
\usepackage{amsmath,amsthm,amsfonts,amssymb,amscd, fancyhdr, color, comment, graphicx, environ}
\usepackage{float}
\usepackage{booktabs}
\usepackage{pifont}
\usepackage{mathrsfs}
\usepackage[math-style=ISO]{unicode-math}
\usepackage{lastpage}
\usepackage[dvipsnames]{xcolor}
\usepackage[framemethod=TikZ]{mdframed}
\usepackage{enumerate}
\usepackage[shortlabels]{enumitem}
\usepackage{fancyhdr}
\usepackage{indentfirst}
\usepackage{listings}
\usepackage{sectsty}
\usepackage{thmtools}
\usepackage{shadethm}
\usepackage{hyperref}
\usepackage{setspace}
\usepackage{adjustbox}
\hypersetup{
    colorlinks=true,
    linkcolor=blue,
    filecolor=magenta,      
    urlcolor=blue,
}
\usepackage{xcolor,colortbl}
%%%%%%%%%%%%%%%%%%%%%%%%%%%%%%%%%%%%%%%%%%%%%%%%%%%%%%%%%%%%%%%%%%
%%%%%%%%%%%%%%%%%%%%%%%%%%%%%%%%%%%%%%%%%%%%%%%%%%%%%%%%%%%%%%%%%%
%Environment setup
\mdfsetup{skipabove=\topskip,skipbelow=\topskip}
\newrobustcmd\ExampleText{%
An \textit{inhomogeneous linear} differential equation has the form
\begin{align}
L[v ] = f,
\end{align}
where $L$ is a linear differential operator, $v$ is the dependent
variable, and $f$ is a given non−zero function of the independent
variables alone.
}
\mdfdefinestyle{theoremstyle}{%
linecolor=black,linewidth=1pt,%
frametitlerule=true,%
frametitlebackgroundcolor=gray!20,
innertopmargin=\topskip,
}
\mdtheorem[style=theoremstyle]{Problem}{Question Number}
\setcounter{Problem}{3}
\newenvironment{Solution}{\textbf{Solution.}}

\definecolor{codegreen}{rgb}{0,0.6,0}
\definecolor{codegray}{rgb}{0.5,0.5,0.5}
\definecolor{codepurple}{rgb}{0.58,0,0.82}
\definecolor{backcolour}{rgb}{0.95,0.95,0.92}

\lstdefinestyle{mystyle}{
    backgroundcolor=\color{backcolour},   
    commentstyle=\color{codegreen},
    keywordstyle=\color{magenta},
    numberstyle=\tiny\color{codegray},
    stringstyle=\color{codepurple},
    basicstyle=\ttfamily\footnotesize,
    breakatwhitespace=false,         
    breaklines=true,                 
    captionpos=b,                    
    keepspaces=true,                 
    numbers=left,                    
    numbersep=5pt,                  
    showspaces=false,                
    showstringspaces=false,
    showtabs=false,                  
    tabsize=2
}

\lstset{style=mystyle}
%%%%%%%%%%%%%%%%%%%%%%%%%%%%%%%%%%%%%%%%%%%%%%%%%%%%%%%%%%%%%%%%%%
%%%%%%%%%%%%%%%%%%%%%%%%%%%%%%%%%%%%%%%%%%%%%%%%%%%%%%%%%%%%%%%%%%
%Fill in the appropriate information below
\newcommand{\norm}[1]{\left\lVert#1\right\rVert}     
\newcommand\course{XXXX0000}                            % <-- course name   
\newcommand\hwnumber{0}                                 % <-- homework number
\newcommand\Information{Someone}                        % <-- personal information
%%%%%%%%%%%%%%%%%%%%%%%%%%%%%%%%%%%%%%%%%%%%%%%%%%%%%%%%%%%%%%%%%%
%%%%%%%%%%%%%%%%%%%%%%%%%%%%%%%%%%%%%%%%%%%%%%%%%%%%%%%%%%%%%%%%%%
%Page setup
\pagestyle{fancy}
\headheight 35pt
\lhead{\today \hspace*{4cm} Key-Breakers}
\rhead{\includegraphics[width=1.2cm]{./logo.png}}
\lfoot{}
\pagenumbering{arabic}
\cfoot{\small\thepage}
\rfoot{}
\headsep 1.2em
\renewcommand{\baselinestretch}{1.25}
%%%%%%%%%%%%%%%%%%%%%%%%%%%%%%%%%%%%%%%%%%%%%%%%%%%%%%%%%%%%%%%%%%
%%%%%%%%%%%%%%%%%%%%%%%%%%%%%%%%%%%%%%%%%%%%%%%%%%%%%%%%%%%%%%%%%%
%Add new commands here
\renewcommand{\labelenumi}{\alph{enumi})}
\newcommand{\Z}{\mathbb Z}
\newcommand{\R}{\mathbb R}
\newcommand{\Q}{\mathbb Q}
\newcommand{\NN}{\mathbb N}
\newcommand{\PP}{\mathbb P}
\DeclareMathOperator{\Mod}{Mod} 
\renewcommand\lstlistingname{Algorithm}
\renewcommand\lstlistlistingname{Algorithms}
\def\lstlistingautorefname{Alg.}
\newtheorem*{theorem}{Theorem}
\newtheorem*{lemma}{Lemma}
\newtheorem{case}{Case}
\newcommand{\assign}{:=}
\newcommand{\infixiff}{\text{ iff }}
\newcommand{\nobracket}{}
\newcommand{\backassign}{=:}
\newcommand{\tmmathbf}[1]{\ensuremath{\boldsymbol{#1}}}
\newcommand{\tmop}[1]{\ensuremath{\operatorname{#1}}}
\newcommand{\tmtextbf}[1]{\text{{\bfseries{#1}}}}
\newcommand{\tmtextit}[1]{\text{{\itshape{#1}}}}

\newenvironment{itemizedot}{
    \begin{itemize} 
    \renewcommand{\labelitemi}{$\bullet$}
    \renewcommand{\labelitemii}{$\bullet$}
    \renewcommand{\labelitemiii}{$\bullet$}
    \renewcommand{\labelitemiv}{$\bullet$}}
    {\end{itemize}}

\catcode`\<=\active\def<{
\fontencoding{T1}\selectfont\symbol{60}\fontencoding{\encodingdefault}}
\catcode`\>=\active\def>{
\fontencoding{T1}\selectfont\symbol{62}\fontencoding{\encodingdefault}}
\catcode`\<=\active\def<{
\fontencoding{T1}\selectfont\symbol{60}\fontencoding{\encodingdefault}}

%%%%%%%%%%%%%%%%%%%%%%%%%%%%%%%%%%%%%%%%%%%%%%%%%%%%%%%%%%%%%%%%%%
%%%%%%%%%%%%%%%%%%%%%%%%%%%%%%%%%%%%%%%%%%%%%%%%%%%%%%%%%%%%%%%%%%
%Begin now!



\begin{document}
%%%%%%%%%%%%%%%%%%%%%%%%%%%%%%%%%%%%%%%%%%%%%%%%%%%%%%%%%%%%%%%%%%
%%%%%%%%%%%%%%%%%%%%%%%%%%%%%%%%%%%%%%%%%%%%%%%%%%%%%%%%%%%%%%%%%%
%Start the assignment now
%%%%%%%%%%%%%%%%%%%%%%%%%%%%%%%%%%%%%%%%%%%%%%%%%%%%%%%%%%%%%%%%%%
%New problem
\newpage
\begin{Problem}

Use the random Sbox you generated for the following

\begin{itemize}
\item Write a code to generate its DDT-LAT in your favorite programming language.
\item Submit code in a separate file and show the DDT-LAT in answer script
\item What is the maximum differential probability of your Sbox? Mention the transition(s)
that lead to that.
\item What is the maximum bias your Sbox? Mention the input-output mask(s) that lead to
that.
\item Now use Sage to generate the DDT-LAT for your Sbox.
\item What is the meaning of component function of an Sbox?
\item Enumerate the component functions of your Sbox using Sage.
\item How would you represent your Sbox as a Boolean function using the component functions?
[Hint: There will be four functions each representing one output bit]
\item Ref: http://match.stanford.edu/reference/cryptography/sage/crypto/sbox.html
\end{itemize}

\end{Problem}

\begin{Solution}

\textbf{Python Code to Generate DDT:}

    \begin{lstlisting}[language=Python]
        #Randomly generated sbox

        sbox = [12,15,7,3,5,9,10,2,14,11,6,1,0,4,13,8]
        
        #	Creating matrix whose values are 0
        ans = [[0 for i in range (16)]for j in range(16)]
        
        # Calculate the DDT by finding input and output differences
        for x in range(16):
            for dx in range(16):
                # Compute the output difference for S-box values
                dy = sbox[x] ^ sbox[x ^ dx]
                # Increment the count in DDT table for input-output difference pair (dx, dy)
                ans[dx][dy] += 1
        
        #Printing the matrix
        for i in ans:
            print(i)

    \end{lstlisting}

    \textbf{Python Code to Generate LAT:}

    \begin{lstlisting}[language=Python]
        import numpy as np

        # Define the S-box
        sbox = [12, 15, 7, 3, 5, 9, 10, 2, 14, 11, 6, 1, 0, 4, 13, 8]
        
        # Define the size of the S-box (4-bit means 16 entries)
        n = 4
        size = 2**n
        
        # Initialize the LAT (Linear Approximation Table)
        lat = np.zeros((size, size), dtype=int)
        
        # Compute the LAT
        for a in range(size):  
            for b in range(size):  
                count = 0
        
                for x in range(size): 
                    
                    # Dot product (XOR) between input mask 'a' and input 'x'
                    input_mask = bin(a & x).count('1') % 2
        
                    # Dot product (XOR) between output mask 'b' and S-box output 'sbox[x]'
                    output_mask = bin(b & sbox[x]).count('1') % 2
        
                    # Check if input_mask == output_mask
                    if input_mask == output_mask:
                        count += 1
                # Populate the LAT with the biased result
        
                lat[a, b] = count - (size // 2)
        
        # Print the LAT table
        print("Linear Approximation Table (LAT):")
        print(lat)
        

    \end{lstlisting}

\textbf{DDT and Max Differential Probability}    

    \begin{lstlisting}
        Differential Distribution Table:
        [16, 0, 0, 0, 0, 0, 0, 0, 0, 0, 0, 0, 0, 0, 0, 0]
        [0, 0, 0, 2, 4, 4, 0, 2, 2, 0, 0, 0, 2, 0, 0, 0]
        [0, 0, 0, 0, 0, 0, 0, 0, 2, 0, 2, 4, 4, 2, 0, 2]
        [0, 0, 0, 2, 0, 0, 0, 2, 4, 2, 0, 0, 0, 2, 0, 4]
        [0, 2, 0, 0, 0, 0, 2, 0, 0, 4, 0, 2, 0, 2, 2, 2]
        [0, 0, 0, 0, 0, 4, 0, 0, 0, 2, 4, 2, 2, 0, 2, 0]
        [0, 0, 2, 4, 0, 2, 4, 0, 0, 0, 2, 0, 0, 2, 0, 0]
        [0, 2, 2, 0, 0, 2, 6, 0, 0, 0, 0, 0, 0, 0, 4, 0]
        [0, 2, 4, 0, 2, 2, 0, 2, 0, 0, 2, 0, 0, 2, 0, 0]
        [0, 4, 2, 0, 0, 2, 2, 2, 0, 2, 0, 0, 0, 0, 0, 2]
        [0, 2, 0, 0, 0, 0, 2, 0, 4, 2, 4, 0, 0, 0, 2, 0]
        [0, 0, 2, 0, 2, 0, 0, 0, 0, 2, 0, 0, 2, 6, 2, 0]
        [0, 0, 2, 2, 0, 0, 0, 0, 0, 0, 2, 6, 4, 0, 0, 0]
        [0, 0, 0, 0, 2, 0, 0, 2, 2, 0, 0, 2, 0, 0, 4, 4]
        [0, 2, 0, 2, 2, 0, 0, 6, 2, 2, 0, 0, 0, 0, 0, 0]
        [0, 2, 2, 4, 4, 0, 0, 0, 0, 0, 0, 0, 2, 0, 0, 2]
        
        Transition(s) Leading to Maximum Differential Probability:
        
        Maximum Value Identification: Scanning the matrix, the maximum value is 6.
        
        Finding Transitions: The value 6 appears at the following positions in the matrix:
            Row 7, Column 6
            Row 12, Column 11
        
        Therefore, the transitions leading to the maximum differential probability of 6 are:
            Input difference 7 to output difference 6
            Input difference 12 to output difference 11
        
        In summary, the maximum differential probability of the S-box is 6/16, and the transitions that lead to this probability are:
        
            Input difference 7 to output difference 6
            Input difference 12 to output difference 11
        
        Maximum Differential Probability:
        
        The Maximum Differential Probability (MDP) is:
        MDP=6/16 =0.375
        MDP=6/16 =0.375
        
        This is the highest differential probability for the S-box.
        Conclusion:
        
        The maximum differential probability for the S-box is 0.375, and it is achieved by the transitions:
        
            Input difference 7 → Output difference 6
            Input difference 12 → Output difference 11.
        

    \end{lstlisting}

    \textbf{LAT and Max Bias}    

    \begin{lstlisting}
                
        Linear Approximation Table (LAT):
        [[ 8  0  0  0  0  0  0  0  0  0  0  0  0  0  0  0]
         [ 0  2  0 -2 -4  2  0  2  0 -2  0  2  0  2  4  2]
         [ 0  0  2  2 -2 -2  0  0 -2  2  4  0  0  4 -2  2]
         [ 0  2  2  0  2  0  0  2  2 -4  0 -2  4  2 -2  0]
         [ 0 -2 -4  2 -2 -4  2  0  0 -2  0 -2  2  0  2  0]
         [ 0  4  0  0 -2 -2  2 -2  0  0 -4  0 -2  2 -2 -2]
         [ 0  2 -2  0  0  2 -2  0 -6  0  0 -2  2  0  0 -2]
         [ 0  0  2 -2  0 -4 -2 -2 -2 -2  0  4  2 -2  0  0]
         [ 0 -2 -2  0  0  2 -2 -4  0 -2 -2  0  0  2 -2  4]
         [ 0  0  2  2  0  0 -2 -2  0 -4  2 -2 -4  0  2 -2]
         [ 0 -2  4 -2 -2  0  2  0 -2  0 -2 -4  0 -2  0  2]
         [ 0  0  0  0 -2 -2 -6  2  2  2 -2 -2  0  0  0  0]
         [ 0  0  2  6 -2  2  0  0  0  0 -2  2  2 -2  0  0]
         [ 0 -2  2  0  2  0  0 -2  0  2 -2  0  2  4  4 -2]
         [ 0  4  0  0  0  0  0 -4  2  2  2 -2  2 -2  2  2]
         [ 0  2  0  2  4 -2  0  2 -2  0 -2  0 -2  0  2  4]]
        
        Maximum Bias: The maximum absolute value in the LAT is 6 (excluding LAT[0][0]). It occurs in multiple positions:
            LAT[6][8] = -6
            LAT[11][6] = -6
            LAT[12][3] = 6
        
        Maximum Bias Calculation: The maximum bias can be expressed as a fraction of the total number of inputs:
        Bias=6/16=0.375
        
        Input-Output Mask(s) Leading to Maximum Bias: The maximum bias of 6 is achieved with the following input-output mask combinations:
            Input mask 6 → Output mask 8
            Input mask 11 → Output mask 6
            Input mask 12 → Output mask 3
        
        Conclusion:
        
        The maximum bias of your S-box is 0.375, and it is achieved by the following input-output mask combinations:
        
            Input mask 6 → Output mask 8
            Input mask 11 → Output mask 6
            Input mask 12 → Output mask 3.
    \end{lstlisting}

\end{Solution}

%%%%%%%%%%%%%%%%%%%%%%%%%%%%%%%%%%%%%%%%%%%%%%%%%%%%%%%%%%%%%%%%%%
%Complete the assignment now
\end{document}

%%%%%%%%%%%%%%%%%%%%%%%%%%%%%%%%%%%%%%%%%%%%%%%%%%%%%%%%%%%%%%%%%%
%%%%%%%%%%%%%%%%%%%%%%%%%%%%%%%%%%%%%%%%%%%%%%%%%%%%%%%%%%%%%%%%%%