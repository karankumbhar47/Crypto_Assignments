
\noindent The Midori64 cipher is a lightweight block cipher designed for
energy-efficient applications. It employs a 4-bit S-box in its substitution
layer, which plays a critical role in ensuring the cipher's security. The S-box
used in Midori64, along with its Differential Distribution Table (DDT), is
analyzed to evaluate its resistance to differential cryptanalysis.

\subsection*{Midori64 S-box}
\noindent The S-box used in the Midori64 cipher is shown below:

\begin{table}[h!]
	\centering
	\setlength{\tabcolsep}{8pt}
	\renewcommand{\arraystretch}{1.5}
	\begin{tabular}{|c|c|c|c|c|c|c|c|c|c|c|c|c|c|c|c|c|}
\hline
x & 0 & 1 & 2 & 3 & 4 & 5 & 6 & 7 & 8 & 9 & A & B & C & D & E & F \\
\hline
S(x) & C & A & D & 3 & E & B & F & 7 & 8 & 9 & 1 & 5 & 0 & 2 & 4 & 6 \\
\hline
\end{tabular} % Generated from Python
	\hspace{2pt}
	\caption{Midori64 S-box}
\end{table}

\subsection{Differential Distribution Table (DDT)}
\noindent The Differential Distribution Table (DDT) of the S-box is presented
below. It quantifies the probability of each output difference occurring for a
given input difference.

\begin{table}[h!]
	\centering
	\label{tab:ddt}
	\caption{Differential Distribution Table (DDT)}
	\setlength{\tabcolsep}{8pt}
	\hspace{2pt}
	\begin{tabular}{c|cccccccccccccccc}
in /\ out & 0 & 1 & 2 & 3 & 4 & 5 & 6 & 7 & 8 & 9 & A & B & C & D & E & F \\
\hline
0 & 16 & - & - & - & - & - & - & - & - & - & - & - & - & - & - & - \\
1 & - & 2 & 4 & - & 2 & 2 & 2 & - & 2 & - & - & - & - & - & 2 & - \\
2 & - & 4 & - & - & 4 & - & - & - & - & 4 & - & - & 4 & - & - & - \\
3 & - & - & - & - & 2 & - & 4 & 2 & 2 & 2 & - & - & - & 2 & - & 2 \\
4 & - & 2 & 4 & 2 & 2 & 2 & - & - & 2 & - & - & 2 & - & - & - & - \\
5 & - & 2 & - & - & 2 & - & - & 4 & - & 2 & 4 & - & 2 & - & - & - \\
6 & - & 2 & - & 4 & - & - & - & 2 & 2 & - & - & - & 2 & 2 & - & 2 \\
7 & - & - & - & 2 & - & 4 & 2 & - & - & - & - & 2 & - & 4 & 2 & - \\
8 & - & 2 & - & 2 & 2 & - & 2 & - & - & 2 & - & 2 & 2 & - & 2 & - \\
9 & - & - & 4 & 2 & - & 2 & - & - & 2 & 2 & - & 2 & 2 & - & - & - \\
A & - & - & - & - & - & 4 & - & - & - & - & 4 & - & - & 4 & - & 4 \\
B & - & - & - & - & 2 & - & - & 2 & 2 & 2 & - & 4 & - & 2 & - & 2 \\
C & - & - & 4 & - & - & 2 & 2 & - & 2 & 2 & - & - & 2 & - & 2 & - \\
D & - & - & - & 2 & - & - & 2 & 4 & - & - & 4 & 2 & - & - & 2 & - \\
E & - & 2 & - & - & - & - & - & 2 & 2 & - & - & - & 2 & 2 & 4 & 2 \\
F & - & - & - & 2 & - & - & 2 & - & - & - & 4 & 2 & - & - & 2 & 4 \\
\end{tabular} % Generated from Python
\end{table}

\newpage
\textbf{Value Counts in the DDT:}

The distribution of values in the Difference Distribution Table (DDT) is as follows:
\begin{table}[h!]
	\centering
	\setlength{\tabcolsep}{5pt}
	\begin{tabular}{|c|c|}
		\hline
		\textbf{Value} & \textbf{Count} \\ \hline
		0              & 159            \\ \hline
		2              & 72             \\ \hline
		4              & 24             \\ \hline
		16             & 1              \\ \hline
	\end{tabular}
\end{table}

\subsection{Differential Distribution Table (DDT) Analysis}

\subsubsection*{1. Differential Uniformity}
\begin{itemize}
	\item \textbf{Observation:} The maximum value in the DDT is 4, which is the
	      largest count (excluding the first row). This means that the
	      \textit{differential uniformity} is \( \Delta_{\text{max}} = 4 \).

	\item \textbf{Conclusion:} The differential uniformity of 4 indicates that
	      the cipher is somewhat resistant to differential cryptanalysis, but it is
	      not as strong as S-boxes with \( \Delta_{\text{max}} = 4 \) (e.g., AES)
	      which occurs only once in row.  As in Midori cipher's DDT contains  \(
	      \Delta_{\text{max}} = 4 \) more than one for the row.  However, for
	      lightweight ciphers like Midori, this trade-off is acceptable, as it
	      balances security with efficiency.
\end{itemize}

\subsubsection*{2. Uniform Distribution of Values}
\begin{itemize}
	\item
	      \textbf{Observation:} The values in the DDT are relatively evenly distributed,
	      and there are no particular input-output differential pairs that dominate the
	      table.
	\item
	      \textbf{Conclusion:} This suggests that the S-box does not have significant
	      biases or vulnerabilities that would make it easier for attackers to exploit
	      specific differentials. This balanced distribution enhances the cipher's
	      resistance to differential cryptanalysis.
\end{itemize}

\subsubsection*{3. Null Entries in the DDT}
\begin{itemize}
	\item
	      \textbf{Observation:} Several entries in the DDT are 0, which means
	      that certain input differences do not produce specific output
	      differences. This occurs in rows like the first and others.
	\item
	      \textbf{Conclusion:} Null entries limit an attacker’s ability to craft
	      specific differential trails, as they indicate that some differentials
	      cannot happen. This property adds an extra layer of security to the
	      S-box.
\end{itemize}

\subsubsection*{4. Symmetry}
\begin{itemize}
	\item
	      \textbf{Observation:} The DDT is symmetric, i.e., if \( D(x, y) \) is an entry,
	      then \( D(y, x) \) is also an entry.
	\item
	      \textbf{Conclusion:} This symmetry is expected for well-designed S-boxes and
	      confirms the correctness of the DDT, ensuring that the cryptographic function
	      behaves consistently in both directions (input to output and vice versa).
\end{itemize}

\subsubsection*{5. Compliance with DDT Properties}
The following propositions hold true for the DDT of Midori64:
\begin{itemize}
	\item \textbf{Proposition 1:} Every entry in the DDT is a non-negative even
	      integer between 0 and \( 2^n \).
	\item \textbf{Proposition 2:} The top-left entry of the DDT is \( 2^n \).
	\item \textbf{Proposition 3:} The first row consists of all zeros except the
	      first entry, which is \( 2^n \).
	\item \textbf{Proposition 4:} The first column consists of all zeros except
	      the first entry, which is \( 2^n \).
	\item \textbf{Proposition 5:} The sum of the entries of each row is \( 2^n \).
	\item \textbf{Proposition 6:} If the S-box is bijective, then every row and
	      column of the DDT adds up to \( 2^n \).
\end{itemize}

The DDT of the Midori64 S-box adheres to these properties, which confirms that
the S-box is well-structured and complies with the expected cryptographic
behaviors.

\subsubsection*{Comparative Analysis with Other Ciphers}
\begin{table}[h!]
	\centering
	\renewcommand{\arraystretch}{1.5} % Adjust row height
	\setlength{\tabcolsep}{5pt} % Adjust column spacing
	\caption{Comparison of DDT Properties Across Ciphers}
	\label{tab:ddt_comparison}
	\resizebox{\textwidth}{!}{%
		\begin{tabular}{|c|c|c|c|c|c|c|}
			\hline
			\textbf{Cipher Name} & \textbf{Number of 0's} & \textbf{Number of 2's} & \textbf{Number of 4's} & \textbf{Number of 16's} & \textbf{Differential Uniformity} \\ \hline
			Pyjamask-128         & -                      & -                      & -                      & -                       & -                                \\ \hline
			Square               & 159                    & 72                     & 24                     & 1                       & 4                                \\ \hline
			PRESENT              & -                      & -                      & -                      & -                       & -                                \\ \hline
			PRINCE               & -                      & -                      & -                      & -                       & -                                \\ \hline
			KLEIN                & -                      & -                      & -                      & -                       & -                                \\ \hline
			LED                  & -                      & -                      & -                      & -                       & -                                \\ \hline
			BACKSHEESH           & 198                    & 0                      & 56                     & 2                       & 4                                \\ \hline
			Midori               & 159                    & 72                     & 24                     & 1                       & 4                                \\ \hline
			GIFT                 & -                      & -                      & -                      & -                       & -                                \\ \hline
			PRINTcipher          & -                      & -                      & -                      & -                       & -                                \\ \hline
			Rectangle            & -                      & -                      & -                      & -                       & -                                \\ \hline
		\end{tabular}%
	}
\end{table}



\subsubsection*{Conclusion}

The Midori64 S-box offers a reasonable trade-off between security and
efficiency, making it suitable for lightweight cryptographic applications. While
it does not have the optimal differential uniformity of \( \Delta_{\text{max}} = 2 \)
seen in AES, it remains secure against differential cryptanalysis and other
attacks due to its balanced distribution and structural properties.

\subsection{Linear Approximation Table (LAT) Analysis of the Midori64 S-box}

\noindent The Linear Approximation Table (LAT) for the S-box is a key component
for analyzing the resistance of the S-box to linear cryptanalysis. It quantifies
how well the S-box behaves under linear approximations, with lower values
indicating better resistance. The LAT for the Midori64 S-box is shown below:

\begin{table}[h!]
	\centering
	\label{tab:lat}
	\caption{Linear Approximation Table (LAT) for the Midori64 S-box}
	\setlength{\tabcolsep}{8pt}
	\vspace{12pt}
	    \begin{tabular}{|c|cccccccccccccccc|}
        \hline
          & 0 & 1 & 2 & 3 & 4 & 5 & 6 & 7 & 8 & 9 & a & b & c & d & e & f \\
        \hline
        0 & 8 & . & . & . & . & . & . & . & . & . & . & . & . & . & . & . \\
        1 & . & 2 & 4 & 2 & -2 & . & 2 & . & -2 & . & 2 & . & 4 & -2 & . & -2 \\
        2 & . & 4 & . & . & 4 & . & . & . & -4 & . & . & . & . & 4 & . & . \\
        3 & . & 2 & . & 2 & -2 & . & 2 & 4 & 2 & -4 & -2 & . & . & 2 & . & 2 \\
        4 & . & -2 & 4 & -2 & 2 & . & -2 & . & -2 & -4 & -2 & . & . & -2 & . & 2 \\
        5 & . & . & . & . & . & . & . & . & . & . & -4 & -4 & . & . & 4 & -4 \\
        6 & . & 2 & . & 2 & -2 & . & 2 & -4 & -2 & . & -2 & . & -4 & -2 & . & 2 \\
        7 & . & . & . & 4 & . & . & -4 & . & . & . & . & -4 & . & . & -4 & . \\
        8 & . & -2 & -4 & 2 & -2 & . & -2 & . & -4 & -2 & . & 2 & 2 & . & 2 & . \\
        9 & . & . & . & -4 & -4 & . & . & . & -2 & 2 & -2 & -2 & 2 & 2 & -2 & 2 \\
        a & . & 2 & . & -2 & -2 & -4 & -2 & . & . & -2 & 4 & -2 & -2 & . & 2 & . \\
        b & . & . & . & . & . & -4 & . & -4 & 2 & -2 & -2 & 2 & 2 & 2 & -2 & -2 \\
        c & . & 4 & . & . & . & . & -4 & . & 2 & 2 & -2 & 2 & 2 & -2 & 2 & 2 \\
        d & . & -2 & 4 & 2 & -2 & . & -2 & . & . & 2 & . & 2 & -2 & 4 & 2 & . \\
        e & . & . & . & . & . & 4 & . & -4 & 2 & -2 & 2 & -2 & 2 & 2 & 2 & 2 \\
        f & . & -2 & . & 2 & 2 & -4 & 2 & . & . & 2 & . & -2 & 2 & . & 2 & 4 \\
        \hline
    \end{tabular}
\end{table}

\textbf{Value Counts in the Linear Approximation Table (LAT)}
This table represents the counts of each value observed in the Linear
Approximation Table (LAT) for the Midori64 S-box.

\begin{table}[h!]
	\centering
	\setlength{\tabcolsep}{5pt}
	\begin{tabular}{|c|c|}
		\hline
		\textbf{Value} & \textbf{Count} \\
		\hline
		-4             & 21             \\
		-2             & 44             \\
		0              & 123            \\
		2              & 52             \\
		4              & 15             \\
		8              & 1              \\
		\hline
	\end{tabular}
\end{table}

\subsection*{LAT Analysis}

\subsubsection*{1. Linear Bias and Distribution of Values}
\begin{itemize}
	\item \textbf{Observation:} The values in the LAT table are spread between
	      positive and negative numbers. A few entries show larger values such as $\pm
		      4$ and $\pm 2$, which represent the linear bias of the approximation. Most
	      values are closer to 0, indicating weaker linear approximations.
	\item \textbf{Conclusion:} The distribution of values suggests that the
	      S-box of Midori64 is somewhat resistant to linear cryptanalysis. Ideally,
	      the values should be close to 0 for most of the entries, which is the case
	      here, though some larger biases may still indicate potential
	      vulnerabilities.
\end{itemize}

\subsubsection*{2. Symmetry in LAT}
\begin{itemize}
	\item \textbf{Observation:} The LAT is symmetric, i.e., the value at
	      position $(i, j)$ is equal to the value at position $(j, i)$, which is
	      expected from a well-designed S-box.
	\item \textbf{Conclusion:} The symmetry of the LAT confirms that the
	      Midori64 S-box behaves consistently in both directions (input to output and
	      vice versa), which is a desirable property for cryptographic functions and
	      indicates correctness in design.
\end{itemize}

\subsubsection*{3. Even Integer Entries in LAT}
\begin{itemize}
	\item \textbf{Observation:} All the entries in the LAT are even integers, as
	      expected from a bijective S-box. This property holds true for the Midori64
	      S-box, which confirms that the S-box is a bijection.
	\item \textbf{Conclusion:} The fact that all LAT entries are even integers,
	      and the first row and column (except the upper-left entry) consist of zeros,
	      is a strong indicator that the S-box is bijective. This makes the S-box more
	      secure against linear cryptanalysis.
\end{itemize}

\subsubsection*{4. Compliance with LAT Properties}
\begin{itemize}
	\item \textbf{Observation:} The first row and the first column are zeros,
	      except for the upper-left entry, which is $2^{n-1}$, indicating that the
	      S-box follows the LAT properties of bijective S-boxes.
	\item \textbf{Conclusion:} The Midori64 S-box adheres to the properties
	      expected from a bijective S-box. This is a positive feature, enhancing its
	      cryptographic strength by ensuring that no biases are introduced through
	      linear approximations.
\end{itemize}


\subsection*{Comparison with Other Ciphers}

\begin{table}[h!]
	\centering
	\renewcommand{\arraystretch}{1.5} % Adjust row height
	\setlength{\tabcolsep}{5pt} % Adjust column spacing
	\caption{Comparison of LAT Properties Across Ciphers}
	\label{tab:ddt_comparison}
	\resizebox{\textwidth}{!}{%
		\begin{tabular}{|c|c|c|c|c|c|c|}
			\hline
			\textbf{Cipher Name} & \textbf{Number of -4's} & \textbf{Number of -2's} & \textbf{Number of 0's} & \textbf{Number of 2's} & \textbf{Number of 4's} & \textbf{Number of 8's} \\ \hline
			Pyjamask-128         & 18                      & 52                      & 122                    & 44                     & 19                     & 1                      \\ \hline
			Square               & -                       & -                       & -                      & -                      & -                      & -                      \\ \hline
			PRESENT              & -                       & -                       & -                      & -                      & -                      & -                      \\ \hline
			PRINCE               & -                       & -                       & -                      & -                      & -                      & -                      \\ \hline
			KLEIN                & -                       & -                       & -                      & -                      & -                      & -                      \\ \hline
			LED                  & -                       & -                       & -                      & -                      & -                      & -                      \\ \hline
			BACKSHEESH           & 30                      & 0                       & 198                    & 0                      & 26                     & 2                      \\ \hline
			Midori               & -21                     & 44                      & 123                    & 52                     & 15                     & 1                      \\ \hline
			PUFFIN               & -                       & -                       & -                      & -                      & -                      & -                      \\ \hline
			GIFT                 & -                       & -                       & -                      & -                      & -                      & -                      \\ \hline
			PRINTcipher          & -                       & -                       & -                      & -                      & -                      & -                      \\ \hline
			Rectangle            & -                       & -                       & -                      & -                      & -                      & -                      \\ \hline
		\end{tabular}%
	}
\end{table}

\subsection*{Conclusion}

\noindent The Midori64 S-box strikes a balance between security and efficiency,
making it suitable for lightweight cryptographic applications. Although its LAT
does show some biases, the values are relatively small, and the overall
resistance to linear cryptanalysis is reasonable. The bijective nature of the
S-box, along with its symmetry and even integer entries, further contributes to
its security. While the Midori64 S-box may not be as optimal as the AES S-box in
terms of linear approximation resistance, it offers a good trade-off between
security and computational efficiency, which is crucial for lightweight
cryptography.
