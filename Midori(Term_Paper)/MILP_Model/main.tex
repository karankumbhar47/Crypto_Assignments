
%%%%%%%%%%%%%%%%%%%%%%%%%%%%%%%%%%%%%%%%%%%%%%%%%%%%%%%%%%%%%%%%%%
%%%%%%%%%%%%%%%%%%%%%%%%%%%%%%%%%%%%%%%%%%%%%%%%%%%%%%%%%%%%%%%%%%
% \setmathfont{TeX Gyre Termes Math}
%Packages
\documentclass[10pt, a4paper]{article}
\usepackage[top=3cm, bottom=4cm, left=2cm, right=2cm]{geometry}
\usepackage{amsmath,amsthm,amsfonts,amssymb,amscd, fancyhdr, color, comment, graphicx, environ}
\usepackage{float}
\usepackage{booktabs}
\usepackage{pifont}
\usepackage{mathrsfs}
% \usepackage[math-style=ISO]{unicode-math}
\usepackage{lastpage}
\usepackage[dvipsnames]{xcolor}
\usepackage[framemethod=TikZ]{mdframed}
\usepackage{enumerate}
\usepackage[shortlabels]{enumitem}
\usepackage{fancyhdr}
\usepackage{indentfirst}
\usepackage{listings}
\usepackage{sectsty}
\usepackage{thmtools}
\usepackage{shadethm}
\usepackage{hyperref}
\usepackage{setspace}
\usepackage{adjustbox}
\hypersetup{
	colorlinks=true,
	linkcolor=blue,
	filecolor=magenta,
	urlcolor=blue,
}
\usepackage{xcolor,colortbl}
%%%%%%%%%%%%%%%%%%%%%%%%%%%%%%%%%%%%%%%%%%%%%%%%%%%%%%%%%%%%%%%%%%
%%%%%%%%%%%%%%%%%%%%%%%%%%%%%%%%%%%%%%%%%%%%%%%%%%%%%%%%%%%%%%%%%%
%Environment setup
\mdfsetup{skipabove=\topskip,skipbelow=\topskip}
\newrobustcmd\ExampleText{%
	An \textit{inhomogeneous linear} differential equation has the form
	\begin{align}
		L[v ] = f,
	\end{align}
	where $L$ is a linear differential operator, $v$ is the dependent
	variable, and $f$ is a given non−zero function of the independent
	variables alone.
}
\mdfdefinestyle{theoremstyle}{%
	linecolor=black,linewidth=1pt,%
	frametitlerule=true,%
	frametitlebackgroundcolor=gray!20,
	innertopmargin=\topskip,
}
\mdtheorem[style=theoremstyle]{Problem}{Question Number}
\setcounter{Problem}{4}
\newenvironment{Solution}{\textbf{Solution.}}

\definecolor{codegreen}{rgb}{0,0.6,0}
\definecolor{codegray}{rgb}{0.5,0.5,0.5}
\definecolor{codepurple}{rgb}{0.58,0,0.82}
\definecolor{backcolour}{rgb}{0.95,0.95,0.92}

\lstdefinestyle{mystyle}{
	backgroundcolor=\color{backcolour},
	commentstyle=\color{codegreen},
	keywordstyle=\color{magenta},
	numberstyle=\tiny\color{codegray},
	stringstyle=\color{codepurple},
	basicstyle=\ttfamily\footnotesize,
	breakatwhitespace=false,
	breaklines=true,
	captionpos=b,
	keepspaces=true,
	numbers=left,
	numbersep=5pt,
	showspaces=false,
	showstringspaces=false,
	showtabs=false,
	tabsize=2
}

\lstset{style=mystyle}
%%%%%%%%%%%%%%%%%%%%%%%%%%%%%%%%%%%%%%%%%%%%%%%%%%%%%%%%%%%%%%%%%%
%%%%%%%%%%%%%%%%%%%%%%%%%%%%%%%%%%%%%%%%%%%%%%%%%%%%%%%%%%%%%%%%%%
%Fill in the appropriate information below
\newcommand{\norm}[1]{\left\lVert#1\right\rVert}
\newcommand\course{XXXX0000}                            % <-- course name   
\newcommand\hwnumber{4}                                 % <-- homework number
\newcommand\Information{Someone}                        % <-- personal information
%%%%%%%%%%%%%%%%%%%%%%%%%%%%%%%%%%%%%%%%%%%%%%%%%%%%%%%%%%%%%%%%%%
%%%%%%%%%%%%%%%%%%%%%%%%%%%%%%%%%%%%%%%%%%%%%%%%%%%%%%%%%%%%%%%%%%
%Page setup
\pagestyle{fancy}
\headheight 35pt
\lhead{\today \hspace*{4cm} Key-Breakers}
% \rhead{\includegraphics[width=1.2cm]{../images/logo.png}}
\lfoot{}
\pagenumbering{arabic}
\cfoot{\small\thepage}
\rfoot{}
\headsep 1.2em
\renewcommand{\baselinestretch}{1.25}
%%%%%%%%%%%%%%%%%%%%%%%%%%%%%%%%%%%%%%%%%%%%%%%%%%%%%%%%%%%%%%%%%%
%%%%%%%%%%%%%%%%%%%%%%%%%%%%%%%%%%%%%%%%%%%%%%%%%%%%%%%%%%%%%%%%%%
%Add new commands here
\renewcommand{\labelenumi}{\alph{enumi})}
\newcommand{\Z}{\mathbb Z}
\newcommand{\R}{\mathbb R}
\newcommand{\Q}{\mathbb Q}
\newcommand{\NN}{\mathbb N}
\newcommand{\PP}{\mathbb P}
\DeclareMathOperator{\Mod}{Mod}
\renewcommand\lstlistingname{Algorithm}
\renewcommand\lstlistlistingname{Algorithms}
\def\lstlistingautorefname{Alg.}
\newtheorem*{theorem}{Theorem}
\newtheorem*{lemma}{Lemma}
\newtheorem{case}{Case}
\newcommand{\assign}{:=}
\newcommand{\infixiff}{\text{ iff }}
\newcommand{\nobracket}{}
\newcommand{\backassign}{=:}
\newcommand{\tmmathbf}[1]{\ensuremath{\boldsymbol{#1}}}
\newcommand{\tmop}[1]{\ensuremath{\operatorname{#1}}}
\newcommand{\tmtextbf}[1]{\text{{\bfseries{#1}}}}
\newcommand{\tmtextit}[1]{\text{{\itshape{#1}}}}

\newenvironment{itemizedot}{
	\begin{itemize}
		\renewcommand{\labelitemi}{$\bullet$}
		\renewcommand{\labelitemii}{$\bullet$}
		\renewcommand{\labelitemiii}{$\bullet$}
		\renewcommand{\labelitemiv}{$\bullet$}}
		{\end{itemize}}

\catcode`\<=\active\def<{
\fontencoding{T1}\selectfont\symbol{60}\fontencoding{\encodingdefault}}
\catcode`\>=\active\def>{
\fontencoding{T1}\selectfont\symbol{62}\fontencoding{\encodingdefault}}
\catcode`\<=\active\def<{
\fontencoding{T1}\selectfont\symbol{60}\fontencoding{\encodingdefault}}

%%%%%%%%%%%%%%%%%%%%%%%%%%%%%%%%%%%%%%%%%%%%%%%%%%%%%%%%%%%%%%%%%%
%%%%%%%%%%%%%%%%%%%%%%%%%%%%%%%%%%%%%%%%%%%%%%%%%%%%%%%%%%%%%%%%%%
%Begin now!
\begin{document}

\section*{MILP Formulation for One Round}

\subsection*{Variables}
\begin{itemize}
    \item $x_i$: Binary variables representing the input differences for SubCell.
    \item $y_i$: Binary variables representing the output differences for SubCell.
    \item $z_i$: Binary variables representing the input differences for MixColumn.
    \item $w_i$: Binary variables representing the output differences for MixColumn.
    \item $p_0, p_1$: Additional binary variables to handle the differential probability patterns.
\end{itemize}

\subsection*{Objective Function}
The objective is to minimize the number of active S-boxes and maximize the differential probability:
\begin{equation}
    \text{Minimize} \quad \sum (2 \cdot p_0 + 3 \cdot p_1)
\end{equation}

\subsection*{Constraints}
\begin{itemize}
    \item \textbf{XOR Constraints:} For each XOR operation, the following inequalities are applied:
    \begin{equation}
        \begin{aligned}
            & u_1 + u_2 - v \geq 0 \\
            & u_1 - u_2 + v \geq 0 \\
            & -u_1 + u_2 + v \geq 0 \\
            & u_1 + u_2 + v \leq 2
        \end{aligned}
    \end{equation}

    \item \textbf{SubCell Constraints:} The SubCell operation is modeled using linear inequalities derived from the S-box's differential distribution table:
    \begin{equation}
        S - x_i \geq 0 \quad \forall i \quad \text{and} \quad \sum_{i=0}^{3} x_i - S \geq 0
    \end{equation}

    \item \textbf{ShuffleCell Constraints:} The ShuffleCell operation permutes the bits according to a fixed pattern:
    \begin{equation}
        y_i - z_j = 0 \quad \text{for the respective permutations.}
    \end{equation}

    \item \textbf{MixColumn Constraints:} The MixColumn operation applies a matrix multiplication over GF(2). It is represented by introducing intermediate variables and constraints such as:
    \begin{equation}
        \begin{aligned}
            & z_4 + z_8 - t_1 \geq 0 \\
            & z_{12} + t_1 - w_0 \geq 0 \\
            & z_{12} + t_1 + w_0 \leq 2
        \end{aligned}
    \end{equation}
\end{itemize}

\section*{Conclusion}
The MILP model for one round of Midori64 is constructed with constraints that describe the operations within the round. Using this model, one can optimize the search for differential characteristics and analyze the cipher's resistance against differential attacks.


\end{document}

%%%%%%%%%%%%%%%%%%%%%%%%%%%%%%%%%%%%%%%%%%%%%%%%%%%%%%%%%%%%%%%%%%
%%%%%%%%%%%%%%%%%%%%%%%%%%%%%%%%%%%%%%%%%%%%%%%%%%%%%%%%%%%%%%%%%%
