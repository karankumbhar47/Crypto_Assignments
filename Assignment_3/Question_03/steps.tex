\item \textbf{Initial Fifth Round Key}

The initial key is provided as a 4x4 matrix at the start of the fifth round:

\[
    \text{Fifth Round Key : }
    \renewcommand{\arraystretch}{1.5}
    \setlength{\arrayrulewidth}{0.6mm}
    \setlength{\tabcolsep}{0.5em}
    \begin{array}{|>{\centering\arraybackslash}m{2em}|>{\centering\arraybackslash}m{2em}|>{\centering\arraybackslash}m{2em}|>{\centering\arraybackslash}m{2em}|}
        \hline
        \text{0xC3} & \text{0x09} & \text{0xAA} & \text{0x3C} \\
        \hline
        \text{0x69} & \text{0x99} & \text{0x5B} & \text{0xB8} \\
        \hline
        \text{0x2F} & \text{0x63} & \text{0x62} & \text{0xF3} \\
        \hline
        \text{0xF3} & \text{0xC1} & \text{0xE1} & \text{0xC0} \\
        \hline
    \end{array}
\]

\item \textbf{Rotate the Last Column}

The last column is rotated as follows:
\[
    \begin{array}{c@{\hskip 2cm}c}
        \text{Last Column Before Rotation} & \text{Last Column After Rotation} \\[1em]
        \begin{array}{|c|}\n            \hline\n\text{0x3C}            \\ \hline
\text{0xB8}            \\ \hline
\text{0xF3}            \\ \hline
\text{0xC0} \\ \hline\n        \end{array}\n        &
        \begin{array}{|c|}\n            \hline\n\text{0xB8}            \\ \hline
\text{0xF3}            \\ \hline
\text{0xC0}            \\ \hline
\text{0x3C} \\ \hline\n        \end{array}\n    \end{array}\n\]

\item \textbf{Substitute Bytes}

Apply the S-Box substitution to the rotated column:
\[
    \begin{array}{c@{\hskip 2cm}c}
        \text{Last Column Before Substitution} & \text{Last Column After Substitution} \\[1em]
        \begin{array}{|c|}\n            \hline\n\text{0xB8}            \\ \hline
\text{0xF3}            \\ \hline
\text{0xC0}            \\ \hline
\text{0x3C} \\ \hline\n        \end{array}\n        &
        \begin{array}{|c|}\n            \hline\n\text{0x6C}            \\ \hline
\text{0x0D}            \\ \hline
\text{0xBA}            \\ \hline
\text{0xEB} \\ \hline\n        \end{array}\n    \end{array}\n\]
\pagebreak
\item \textbf{Apply Round Constant}

XOR the substituted column with the round constant.

\[
\begin{align*}
    & 
    \begin{array}{|c|}\hline\n\text{0x6C} \\ \hline\n\text{0x0D} \\ \hline\n\text{0xBA} \\ \hline\n\text{0xEB} \\ \hline\n\end{array} 
    \quad \oplus \quad
    \begin{array}{|c|}\hline\n\text{0x20} \\ \hline\n\text{0x00} \\ \hline\n\text{0x00} \\ \hline\n\text{0x00} \\ \hline\n\end{array} 
    \quad = \quad
    \begin{array}{|c|}\hline\n\text{0x4C} \\ \hline\n\text{0x0D} \\ \hline\n\text{0xBA} \\ \hline\n\text{0xEB} \\ \hline\n\end{array}
\end{align*}
\]\item \textbf{Generating the next round key }

The new key word is generated by XORing the resulting column with the previous column.
\[
        \begin{align*}
            & 
            \begin{array}{|c|}\hline\n\text{0xC3} \\ \hline\n\text{0x69} \\ \hline\n\text{0x2F} \\ \hline\n\text{0xF3} \\ \hline\n\end{array} 
            \quad \oplus \quad
            \begin{array}{|c|}\hline\n\text{0x4C} \\ \hline\n\text{0x0D} \\ \hline\n\text{0xBA} \\ \hline\n\text{0xEB} \\ \hline\n\end{array} 
            \quad = \quad
            \begin{array}{|c|}\hline\n\text{0x8F} \\ \hline\n\text{0x64} \\ \hline\n\text{0x95} \\ \hline\n\text{0x18} \\ \hline\n\end{array}
        \end{align*}
        \]\[
        \begin{align*}
            & 
            \begin{array}{|c|}\hline\n\text{0x09} \\ \hline\n\text{0x99} \\ \hline\n\text{0x63} \\ \hline\n\text{0xC1} \\ \hline\n\end{array} 
            \quad \oplus \quad
            \begin{array}{|c|}\hline\n\text{0x8F} \\ \hline\n\text{0x64} \\ \hline\n\text{0x95} \\ \hline\n\text{0x18} \\ \hline\n\end{array} 
            \quad = \quad
            \begin{array}{|c|}\hline\n\text{0x86} \\ \hline\n\text{0xFD} \\ \hline\n\text{0xF6} \\ \hline\n\text{0xD9} \\ \hline\n\end{array}
        \end{align*}
        \]\[
        \begin{align*}
            & 
            \begin{array}{|c|}\hline\n\text{0xAA} \\ \hline\n\text{0x5B} \\ \hline\n\text{0x62} \\ \hline\n\text{0xE1} \\ \hline\n\end{array} 
            \quad \oplus \quad
            \begin{array}{|c|}\hline\n\text{0x86} \\ \hline\n\text{0xFD} \\ \hline\n\text{0xF6} \\ \hline\n\text{0xD9} \\ \hline\n\end{array} 
            \quad = \quad
            \begin{array}{|c|}\hline\n\text{0x2C} \\ \hline\n\text{0xA6} \\ \hline\n\text{0x94} \\ \hline\n\text{0x38} \\ \hline\n\end{array}
        \end{align*}
        \]\[
        \begin{align*}
            & 
            \begin{array}{|c|}\hline\n\text{0x3C} \\ \hline\n\text{0xB8} \\ \hline\n\text{0xF3} \\ \hline\n\text{0xC0} \\ \hline\n\end{array} 
            \quad \oplus \quad
            \begin{array}{|c|}\hline\n\text{0x2C} \\ \hline\n\text{0xA6} \\ \hline\n\text{0x94} \\ \hline\n\text{0x38} \\ \hline\n\end{array} 
            \quad = \quad
            \begin{array}{|c|}\hline\n\text{0x10} \\ \hline\n\text{0x1E} \\ \hline\n\text{0x67} \\ \hline\n\text{0xF8} \\ \hline\n\end{array}
        \end{align*}
        \]\item \textbf{Next round Key(Sixth Round Key)}

\[
    \text{Sixth Round Key : }
    \renewcommand{\arraystretch}{1.5}
    \setlength{\arrayrulewidth}{0.6mm}
    \setlength{\tabcolsep}{0.5em}
    \begin{array}{|>{\centering\arraybackslash}m{2em}|>{\centering\arraybackslash}m{2em}|>{\centering\arraybackslash}m{2em}|>{\centering\arraybackslash}m{2em}|}
        \hline
        \text{0x8F} & \text{0x86} & \text{0x2C} & \text{0x10} \\
        \hline
        \text{0x64} & \text{0xFD} & \text{0xA6} & \text{0x1E} \\
        \hline
        \text{0x95} & \text{0xF6} & \text{0x94} & \text{0x67} \\
        \hline
        \text{0x18} & \text{0xD9} & \text{0x38} & \text{0xF8} \\
        \hline
    \end{array}
\]

